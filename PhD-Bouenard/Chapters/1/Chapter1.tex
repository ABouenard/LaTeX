\chapter{Introduction}
\markboth{Introduction}{Introduction}
\label{chapter:Introduction}


%%%%%%%%%%%%%%%%%%%%%%%%%%%%%%%%%%%%%%%%%%%%%%%%%%%%%%%%%%%%%%%%%%%%%%%%%%%%%%%%%%%%%%%%%%%%%%%%%%%%%%%%%%%%%%%%%%%%

People have always practiced music and built various musical instruments. Most of the history in the development of musical instruments can be seen as a continuous quest for new instrumental experiences. The advent of computers has rapidly made possible to expand sound possibilities, for instance by recording, sampling, transforming and playing back any sound that can be produced by musical instruments. Later, with the everyday use of computer technologies, the last decades have gone beyond the exploration of sound possibilities, namely by focusing on new ways of thinking interfaces and interaction models to control sound production processes. A plethora of software and hardware musical interfaces have then been proposed, with the common aim of expanding musical performance practice to territories previously unaccessible with traditional musical instruments. One of such territory at the heart of this dissertation is the synthesis of musical performances. Such virtual performances involve virtual characters, where both the gestures made by a virtual musician and the resulting sounds generated by its interaction with a virtual musical instrument are simulated. %This thesis addresses the synthesis of musical peformances involving an animated virtual musician controlling sound synthesis processes, with a special interest about percussion performances.

%%%%%%%%%%%%%%%%%%%%%%%%%%%%%%%%%%%%%%%%%%%%%%%%%%%%%%%%%%%%%%%%%%%%%%%%%%%%%%%%%%%%%%%%%%%%%%%%%%%%%%%%%%%%%%%%%%%%


%%%%%%%%%%%%%%%%%%%%%%%%%%%%%%%%%%%%%%%%%%%%%%%%%%%%%%%%%%%%%%%%%%%%%%%%%%%%%%%%%%%%%%%%%%%%%%%%%%%%%%%%%%%%%%%%%%%%

	\section{Interests and Challenges in Synthesizing Percussion Performances}

This dissertation addresses the synthesis of musical situations with a particular focus on percussion performances. These latter are unique among musical performances, both regarding the nature of percussion instruments and the way performers are playing it.\\

It is generally admitted that the act of percussion for making music dates back to the prehistoric era, when hand claps or body strokes were used to convey expressions \citeIPA{rossing00}. Percussion instruments are therefore of special interest due to the fact there is a tiny distinction between the notions of object and instrument. As every object has an inherent acoustic response to everyday actions such as hitting, scraping or slaping, every object can be considered as a percussion instrument \citeCM{cook:NIME01}. For instance, a table could hardly been played as a string instrument, whereas people can easily strike or roll tools on it while feeling and hearing the resulting feedback.\\
%From an acoustical point of view, percussion instruments are also unique since beat impact forces are directly transmitted to the resonator (even sometimes an integral part of the instrument), instead of using intermediate and complex mechanisms such as reeds.

This blured frontier between everyday objects and percussion instruments may explain why a large amount of percussion instruments is nowadays available, and therefore why percussion performers are highly trained to play on a large collection of instruments \citeIPA{rossing00}. During instrumental situations, performers especially involve complex mechanisms for striking a percussion instrument according to a desired sound effect. Interestingly, percussion performers are not only experts in maintaining rhythm, but also in creating a wide range of timbres even from a single percussion object. Supporting evidence for this claim can for instance be heard and seen concerning John Bonham's drum outtakes, or Max Roach's "hi hat" solo. Percussion performers often have to learn to consider percussion instruments with their inherent and subtle timbre capabilities in addition to rhythmic patterns. Such consideration can lead to learning more complex and abstract mechanisms, such as the elaboration of an own understanding of gesture-sound mechanisms. Among them, a noticeable one's is referred to as sound projection, a mechanism by which percussion performers propose an interpretation of the resulting sounds. John Cage explains this phenomenon as the "feeling given by certain blocks of sounds (...) giving a sense of prolongation, a journey into space" \citeIPA{patterson08}. To that mean, percussionists develop an expert knowledge of how to shape their gestures according to the desired sound effects. This involves for instance an accurate control of preparatory gestures and beat impact forces, especially considering the short time duration of stroke impacts \citeIPA{wagner:MsC06}.\\

For all these reasons, synthesizing percussion performances creates challenges in transposing these real-world phenomena into virtual situations. In fact, due to the complex mechanisms occuring during (real) percussion performances, synthesizing percussion performances necessitates the availability of motion data recorded for ensuring the realism of the synthesized percussion motion. Such data are also helpful for conducting a fine analysis of percussion motion under various playing conditions.\\

But what is exactly the interest of synthesizing percussion performances? The analysis of percussion performances may lead to understanding these underlying complex mechanisms. The use of such \emph{a priori} knowledge can be helpful for, on the one hand, guiding the design of a system able to synthesize virtual percussion performances, and on the other hand for its evaluation. The interest of such system is to provide a realistic modeling of percussion performances, as well as to offer a tool for creating novel performances that may go beyond this reality. Pushing the limits of the reality can involve for instance various virtual character models or different interaction schemes between the simulated motion and sound.

Furthermore, a system able to synthesize percussion performances can be used in many musical applications. Among them, one application could consist in providing a pedagogical tool dedicated to percussion professors and students for exploring synthesized percussion performances with both their visual and sounding feedback. Such a tool could also been used in a musical composition context, by offering a user interface where novel percussion performances can be synthesized given various inputs regarding percussion playing conditions at the gestural level. %Another application could involve live percussion performances, mixing real and virtual performers. The latter application could be useful for exploring interaction processes between performers, by defining interaction rules governing the reactions of the virtual performer in response to those of a real performer.

%%%%%%%%%%%%%%%%%%%%%%%%%%%%%%%%%%%%%%%%%%%%%%%%%%%%%%%%%%%%%%%%%%%%%%%%%%%%%%%%%%%%%%%%%%%%%%%%%%%%%%%%%%%%%%%%%%%%


%%%%%%%%%%%%%%%%%%%%%%%%%%%%%%%%%%%%%%%%%%%%%%%%%%%%%%%%%%%%%%%%%%%%%%%%%%%%%%%%%%%%%%%%%%%%%%%%%%%%%%%%%%%%%%%%%%%%

	\section{Research Questions and Contributions}

The modeling and motion control of a percussionist interacting with sound synthesis necessitates to answer the following research questions:

\begin{itemize}
	\item a fine understanding of the motion involved in real music performances. In other words, what are the essential motion characteristics that have to be preserved when synthesizing music performances, especially considering interaction and also preparatory gestures?
	\item a complete modeling of a real performer, especially in the case of physics-based simulations of music performances. In other words, how can one move from simulations involving simple rigid bodies to simulations involving the actions of a physics-based model of a real performer?
	\item a correspondence of the motion control rules for putting the virtual performer into motion, with real motion data. In other words, how can one ensure that the motion control paradigms involved in the animation of a virtual performer are consistent with real motion data?
	\item an interaction scheme between the simulated gestures of a virtual performer with sound synthesis models. In other words, how can one explicitly (at the physics level) control sound processes by the actions of the virtual performer?
\end{itemize}

We propose a system that realizes the synthesis of the visual and sound feedback of percussion performances in which a virtual percussionist controls sound synthesis processes. The synthesis system is conducted by percussion motion data. It lays its foundations on the analysis of percussion movements, as well as on its musical evaluation.\\

The analysis step of our work shows the importance of the fine control of mallet extremity trajectories by expert percussion performers playing timpani. It includes the collection of instrumental gesture data from several percussionists. We extract movement parameters from the recorded mallet extremity trajectories for different percussion playing techniques. Such parameters are quantitatively evaluated with respect to their ability to represent and discriminate the various playing techniques under study \citeIPA{bouenard:ENACTIVE08, bouenard:GW09, bouenard:AAA09}.\\

We then propose a system for synthesizing timpani performances involving the physical modeling of a virtual percussionist that interacts with sound synthesis processes. The physical framework includes a novel scheme for controlling the motion of the virtual percussionist solely by the specification of trajectories of mallet extremities. This control mode is shown to be consistent with the predominant control of mallet extremity presented in the previous analysis step. The physical approach is also used for allowing the virtual percussionist to interact with a physical model of a timpani \citeCGA{bouenard:HAL09, bouenard:CASA09, bouenard:GM09}.\\

Finally, the proposed system is used in a musical performance perspective. A composition process based on gesture scores is proposed to achieve the synthesis of novel percussion performances. Such gesture scores are obtained by the assembly and articulation of gesture units available in the recorded data. This compositional approach is applied to the synthesis of several percussion exercises, and is informally evaluated by a percussion professor \citeCM{bouenard:NIME08, bouenard:ICMC09, bouenard:CMJ09}.

%%%%%%%%%%%%%%%%%%%%%%%%%%%%%%%%%%%%%%%%%%%%%%%%%%%%%%%%%%%%%%%%%%%%%%%%%%%%%%%%%%%%%%%%%%%%%%%%%%%%%%%%%%%%%%%%%%%%


%%%%%%%%%%%%%%%%%%%%%%%%%%%%%%%%%%%%%%%%%%%%%%%%%%%%%%%%%%%%%%%%%%%%%%%%%%%%%%%%%%%%%%%%%%%%%%%%%%%%%%%%%%%%%%%%%%%%

%\section{Research Context}

%This PhD work has been jointly achieved in the SAMSARA/VALORIA\footnote{\href{http://www-valoria.univ-ubs.fr/SAMSARA/index_en.html}{SAMSARA/VALORIA} laboratory, Universit{\'e} de Bretagne Sud, Vannes, France} and IDMIL/CIRMMT\footnote{\href{http://www.idmil.org}{IDMIL/CIRMMT} laboratory, McGill University, Montreal, Qc., Canada} laboratories as part of the \emph{ICASS}\footnote{\href{http://www-valoria.univ-ubs.fr/SAMSARA/projet_iccass_en.html}{\emph{Interaction and Control of Computer Animation and Sound Simulations}} research project} and \emph{PGAS}\footnote{\href{http://www.idmil.org/projects/percussivegestures?id=projects/percussivegestures}{\emph{Percussive Gesture Analysis and Synthesis}} research project} research projects.\\

%The SAMSARA laboratory focuses on a research theme graviting around the interaction and intelligence of computer systems, with three main research problematics. This involves first the study of the gesture/image relationship for analysing and modeling human gestures to propose systems involving gestural interactions and animated virtual characters. Another theme is the modeling, indexing and retrieval of semi-structured data for the management of large multimedia databases. The last problematic focuses on new human-computer interaction techniques for proposing an enhanced communication between users and computers.\\

%The IDMIL laboratory focuses on the interaction of musical gestures with sound synthesis processes, with three main research themes. This includes first the study of performer's musical gestures for proposing novel input and interaction metaphors for human-computer applications. Another problematic is the study and development of novel sensors for monitoring human actions and motion for the design of new musical interfaces. Lastly, it includes the exploration of the gestural control of sound synthesis processes.\\

%In this work, the collaboration between the two laboratories has yielded to a complementary sharing of competences. Percussion motion data have been captured at the IDMIL laboratory. The analysis of these motion data has been conducted in both SAMSARA and IDMIL laboratories. The system for physically animating a virtual percussionist has been developped at the SAMSARA laboratory. Finally, the musical evaluation of the system has been carried out at the IDMIL laboratory.

%%%%%%%%%%%%%%%%%%%%%%%%%%%%%%%%%%%%%%%%%%%%%%%%%%%%%%%%%%%%%%%%%%%%%%%%%%%%%%%%%%%%%%%%%%%%%%%%%%%%%%%%%%%%%%%%%%%%


%%%%%%%%%%%%%%%%%%%%%%%%%%%%%%%%%%%%%%%%%%%%%%%%%%%%%%%%%%%%%%%%%%%%%%%%%%%%%%%%%%%%%%%%%%%%%%%%%%%%%%%%%%%%%%%%%%%%

	\section{Thesis Structure}

The reminder of this document is organized as follows.\\

Chapter \ref{chapter:StateOfTheArt} reviews related works, covering each module that is necessary for proposing virtual percussion performances involving animated virtual characters that can interact with sound synthesis processes. This includes previous works related to \emph{Computer Animation}, \emph{Computer Music}, as well as percussion-related models.\\

Chapter \ref{chapter:ThesisOverview} presents an overview of the proposed system for synthesizing percussion performances.\\

Chapter \ref{chapter:Analysis} conducts an analysis of motion data recorded from real percussion performances. This analysis leads to the extraction and evaluation of motion parameters for discriminating between various playing conditions.\\

Chapter \ref{chapter:Synthesis} presents and evaluates a complete modeling of a virtual character, as well as a hybrid motion control system dedicated to the simulation of percussion movements. A drum model and the way the virtual percussionist can interact with it are also proposed.\\

Chapter \ref{chapter:Application} conducts experiments to evaluate the musical possibilities of the proposed system in synthesizing novel percussion performances from recorded motion clips.\\

Finally, Chapter \ref{chapter:Conclusion} concludes this document, and proposes future perspectives for extending this work.

%%%%%%%%%%%%%%%%%%%%%%%%%%%%%%%%%%%%%%%%%%%%%%%%%%%%%%%%%%%%%%%%%%%%%%%%%%%%%%%%%%%%%%%%%%%%%%%%%%%%%%%%%%%%%%%%%%%%









